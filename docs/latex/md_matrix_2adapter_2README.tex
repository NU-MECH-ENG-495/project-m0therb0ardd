\chapter{PCB adapter for Raspberry Pi to Hub75 RGB Matrixes}
\hypertarget{md_matrix_2adapter_2README}{}\label{md_matrix_2adapter_2README}\index{PCB adapter for Raspberry Pi to Hub75 RGB Matrixes@{PCB adapter for Raspberry Pi to Hub75 RGB Matrixes}}
Since hand-\/wiring can be a little tedious, here are some PCBs that help with the wiring when using the {\ttfamily rpi-\/rgb-\/led-\/matrix} code.




\begin{DoxyItemize}
\item \href{./passive-3}{\texttt{ Passive-\/3}} Supports three parallel chains, directly connected to the output of a Rapsberry Pi with 40 GPIO pins. Works, but usually it is better to buffer the outputs using the ...
\item \href{./active-3}{\texttt{ Active-\/3}} board. Supports three parallel chains with active buffering and 3.\+3V -\/\texorpdfstring{$>$}{>} 5V level shifting for best reliability. Requires SMD soldering.

As another option you can buy it from these locations not affiliated with this project. They are given to help you locate premade boards but no guarantees are given or implied\+:
\begin{DoxyItemize}
\item \href{https://www.electrodragon.com/product/rgb-matrix-panel-drive-board-raspberry-pi/}{\texttt{ https\+://www.\+electrodragon.\+com/product/rgb-\/matrix-\/panel-\/drive-\/board-\/raspberry-\/pi/}} (\$3/board, but fairly long and/or expensive shipping from HKG). The old board includes support for an optional RTC (real time clock) which had to be disabled for most users who wanted 3 channels instead of an RTC.
\item The new HD board with angled connectors (thinner footprint), is here\+: \href{https://www.electrodragon.com/product/rgb-matrix-panel-drive-board-for-raspberry-pi-v2/}{\texttt{ https\+://www.\+electrodragon.\+com/product/rgb-\/matrix-\/panel-\/drive-\/board-\/for-\/raspberry-\/pi-\/v2/}} \texorpdfstring{$<$}{<}\texorpdfstring{$<$}{<} {\bfseries{this is the recommended board for most users today}} 
\end{DoxyItemize}
\item The \href{./passive-rpi1}{\texttt{ Passive-\/\+RPi1}} adapter board is to connect one panel to Raspberry Pi 1 with 26 GPIO pins.
\item For completeness, Adafruit has a single channel active board here\+: \href{https://www.adafruit.com/product/3211}{\texttt{ https\+://www.\+adafruit.\+com/product/3211}} although it is ultimately inferior to the electrodragon board, but it does ship quicker if you\textquotesingle{}re in the US (note that you will need special compile option or command line argument since it uses non standard wiring)
\item As of 2024/12 and this thread\+: \href{https://rpi-rgb-led-matrix.discourse.group/t/solved-aliexpress-2-string-passive-adapter-not-working-because-rpi-connector-was-soldered-on-the-wrong-side-of-the-board/892/26}{\texttt{ https\+://rpi-\/rgb-\/led-\/matrix.\+discourse.\+group/t/solved-\/aliexpress-\/2-\/string-\/passive-\/adapter-\/not-\/working-\/because-\/rpi-\/connector-\/was-\/soldered-\/on-\/the-\/wrong-\/side-\/of-\/the-\/board/892/26}} , there is also a cheap passive 2 channel board that works great on a small board like r\+Pi0 2W, but all chinese sellers that Marc Merlin surveyed at the time, sold the board soldered wrong and they do not work unless you order the board unsoldered and solder it yourself. Electrodragon is in the process of finishing a low profile 3 channel passive board that will be a proper replacement for the unnamed mis-\/soldered chinese board.
\end{DoxyItemize}

 